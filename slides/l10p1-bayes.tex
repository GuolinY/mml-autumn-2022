%% Time-stamp: <2018-10-18 20:24:12 (marc)>
\documentclass[xcolor=x11names,compress,mathserif,handout]{beamer}

\newcommand{\hackspace}{\hspace{4.2mm}}
\newcommand{\showstudent}[1]{}
\newcommand\hmmax{0}
\newcommand\bmmax{0}


\usepackage{../includes/MarkMathCmds}





% talk/author information
\newcommand{\authorname}{Mark van der Wilk}
\newcommand{\authoremail}{m.vdwilk@imperial.ac.uk}
\newcommand{\authoraffiliation}{
  Department of Computing\\Imperial
  College London}
\newcommand{\authortwitter}{markvanderwilk}
\newcommand{\slidesettitle}{\imperialBlue{Bayesian Inference}}
\newcommand{\footertitle}{Bayesian Inference}
\newcommand{\location}{Imperial College London}
\newcommand{\talkDate}{November 10, 2022}



\date{\imperialGray{\talkDate}}




% load defaults
\input{../includes/header.tex}


\input{../includes/titlepage.tex}
\linespread{1.2}





\begin{frame}{Coin lottery}
Trick question (i.e.~many correct answers):
\onslide*<1>{\begin{itemize}
\item I pick a coin out of my pocket.
\item I flip it 4 times.
\item I observe Heads 4 times.
\item What do you think the chance is of Heads on the next round?
\end{itemize}}
\onslide*<2>{\begin{itemize}
\item I pick a coin out of my pocket.
\item I flip it 10 times.
\item I observe Heads 10 times.
\item What do you think the chance is of Heads on the next round?
\end{itemize}}
\end{frame}


\newcommand{\ph}{p_h}
\begin{frame}{Coin: Maximum Likelihood}
\begin{itemize}
\item Ok, so our probability of heads is unknown. \pause
\item What is our model of the coin? \pause
\begin{gather}
X_i \sim \mathrm{Bernoulli}(\ph) \\
P(X_i = 1) = \ph \qquad \qquad P(X_i = 0) = (1-\ph) \\
\implies P_{X_i}(x) = \ph^x(1-\ph)^{1-x}
\end{gather} \pause
\item How do we find $\ph$? \pause
\item Maximum likelihood?
\end{itemize}
\end{frame}

\begin{frame}{Coin: Maximum Likelihood}
\begin{gather}
p(x_1, x_2, x_3, \dots | \ph) = p(\vx|\ph) = \prod_{n=1}^N P_{X_i}(x_i) \\
\onslide<2->{\log p(\vx|\ph) = \underbrace{\left(\sum_{n=1}^N x\right)}_{N_1} \log \ph + \underbrace{\left(\sum_{n=1}^N(1-x)\right)}_{N_0} \log (1-\ph)} \\
\onslide<3->{\deriv[]{\ph}\log p(\vx|\ph) = 0} \\
\onslide<4->{\implies \ph = \frac{N_1}{N_0 + N_1}}
\end{gather} \pause \pause \pause \pause
\vspace{-0.3cm}
\begin{itemize}
\item After $N_1=4, N_0=0$, would you bet all your savings on heads? \pause
\item Maximum likelihood tells you that you should...
\end{itemize}
\end{frame}

\begin{frame}{Bayesian Inference}
\begin{itemize}
\item Problem: Even after $N_1=4, N_0=0$ you're still uncertain \pause
\item How to quantify your certainty? \pause
\end{itemize}
\begin{center}
\Large Idea: Use probability theory to represent your uncertainty
\end{center} \pause
\begin{itemize}
\item Consider the unknown parameter \emph{unobserved} \pause
\item Data is drawn conditional on parameter \pause
\item Find probability of parameter given the data \pause
\item Use conditional probability (Bayes rule) to quantify your uncertainty!
\end{itemize}
\end{frame}

\begin{frame}{Bayes}
\begin{align}
P(\text{hidden} | \text{data}) = \frac{P(\text{data} | \text{hidden}) p(\text{hidden})}{p(\text{data})}
\end{align} \pause

Coin flipping example.
\end{frame}



\begin{frame}{}










\end{document}
%%% Local Variables: 
%%% mode: latex
%%% TeX-master: t
%%% End: 
