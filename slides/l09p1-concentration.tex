%% Time-stamp: <2018-10-18 20:24:12 (marc)>
\documentclass[xcolor=x11names,compress,mathserif,handout]{beamer}

\newcommand{\hackspace}{\hspace{4.2mm}}
\newcommand{\showstudent}[1]{}
\newcommand\hmmax{0}
\newcommand\bmmax{0}


\usepackage{../includes/MarkMathCmds}





% talk/author information
\newcommand{\authorname}{Mark van der Wilk}
\newcommand{\authoremail}{m.vdwilk@imperial.ac.uk}
\newcommand{\authoraffiliation}{
  Department of Computing\\Imperial
  College London}
\newcommand{\authortwitter}{markvanderwilk}
\newcommand{\slidesettitle}{\imperialBlue{Concentration Inequalities}}
\newcommand{\footertitle}{Overfitting \& Generalisation}
\newcommand{\location}{Imperial College London}
\newcommand{\talkDate}{November 7, 2022}



\date{\imperialGray{\talkDate}}




% load defaults
\input{../includes/header.tex}


\input{../includes/titlepage.tex}
\linespread{1.2}




\begin{frame}[t]{Recap}
Last lecture: Careful mathematical reasoning to \emph{prove} that \pause
\begin{itemize}
\item Loss at deployment converged to \emph{expected loss} as $N\to\infty$ \pause
\item Test set loss converged to \emph{expected loss} as $N\to\infty$ \pause
\item Variance of test set loss scaled as $\frac{c}{N}$\pause
\end{itemize}

\vspace{0.3cm}
Cornerstone of the argument was a \emph{theorem}: Weak LLN:
\begin{align}
\mathbb{P}(|X_n - \mu| < \epsilon) = 1 && \text{for } X_n = \frac{1}{n}\sum_{i=1}^n X_i \,, \{X_n\} \mathrm{iid}\,, \mu = \Exp{}{X_n}
\end{align}
\pause
\vspace{-0.3cm}
\begin{itemize}
\item Doesn't say anything about the \emph{accuracy} for finite $N$! \pause
\item Intuitively, low variance $\implies$ unlikely to be far from mean. \pause
\item Can we use this? Can we make this \emph{precise}?
\end{itemize}
\end{frame}


\begin{frame}{Concentration Inequalities}
\begin{itemize}
\item Theorems are useful because they are \emph{black boxes} \pause
\item Abstract away details of a complex argument, \\ to give you simple answers \pause
\item Today: We break open the black box of the LLN (i.e.~the proof) \pause
\item We find tools that will help us answer questions about finite $N$! \pause
\end{itemize}

\vspace{0.3cm}

Questions:
\begin{enumerate}
\item How accurate is our estimate of the expected loss?
\item How big should our test set be, to get a certain accuracy?
\end{enumerate}
\end{frame}


\section{Intro: Proof of Weak Law of Large Numbers}



\begin{frame}{Weak Law of Large Numbers} 
\begin{itemize}
\item For a sequence of iid RVs $X_1, X_2, X_3, \dots, X_N$ 
\item with mean $\mu = \Exp{}{X}$ 
\item we can define a new RV $\overline{X}_N\ = \frac{1}{N}\sum_{n=1}^N X_n$ 
\item for which will hold:
\begin{align}
\lim_{N\to\infty}\mathbb P\left(|\overline X_n - \mu| < \epsilon\right) = 1
\end{align}
\end{itemize} \pause

\vspace{0.3cm}

\begin{itemize}
\item How to prove this? \pause
\item Let's understand how far samples lie from the mean. \pause
\item For positive RVs, since $|\overline X_n - \mu| \geq 0$!
\end{itemize}
\end{frame}



\begin{frame}[t]{Markov's inequality}
\vspace{-0.6cm}
\begin{myblock}{}
For a RV $X > 0$, and $a > 0$, then
\begin{align}
  P(X \geq a) \leq \frac{\Exp{}{X}}{a}
\end{align}
\end{myblock}\pause

Proof: \pause
\vspace{-0.5cm}
\begin{align}
\Exp{}{X} &= \int_0^\infty x p_X(x) \calcd x \\
\uncover<4->{&= \int_0^a x p_X(x) \calcd x + \int_a^\infty x p_X(x) \calcd x} \\
\uncover<5->{&\geq \int_a^\infty x p_X(x) \calcd x} \\
\uncover<6->{&\geq \int_a^\infty a p_X(x) \calcd x} \\
\uncover<7->{&= a P(X \geq a)} \\
\uncover<8->{\implies& P(X \geq a) \leq \frac{\Exp{}{X}}{a} && \text{Done.}}
\end{align}
\end{frame}


\begin{frame}{Markov's inequality}
For positive RVs (like deviations) with finite means: \pause
\begin{itemize}
\item Large values are increasingly unlikely! ($\propto \frac{1}{a}$) \pause
\item The expectation determines how large values can be \pause
\end{itemize}

\vspace{0.3cm}

Such bounds are powerful because they abstract away details of the distribution, which we may not know! \pause
\end{frame}


\begin{frame}[t]{Chebyshev's Inequality}
\vspace{-0.6cm}
\begin{myblock}{}
For a RV $X$, with finite $\exp{}{X} = \mu$, and finite $\Var{}{X} = \sigma^2$, then for $k>0$
\begin{align}
  P(|X-\mu| \geq k\sigma) \leq \frac{1}{k^2}
\end{align}
\end{myblock}\pause

Proof: Apply Markov's inequality to the RV of the squared deviation: \pause
%\vspace{-0.5cm}
\begin{align}
P\left((X-\mu)^2\geq a\right) &\leq \frac{\Exp{}{(X-\mu)^2}}{a} \\ &= \frac{\sigma^2}{a} \\
\uncover<4->{\implies P\left((X-\mu)^2\geq k^2\sigma^2 \right) &\leq \frac{1}{k^2} && \text{sub }a = k^2\sigma^2} \\
\uncover<5->{\implies P\left(|X-\mu| \geq k\sigma\right) &\leq \frac{1}{k^2} && \text{Done.}}
\end{align}
\end{frame}

\begin{frame}{Chebyshev's Inequality}
For \emph{any} RV with finite mean and variance, we \emph{limit} the probability of being $k$ standard deviations from the mean.
\end{frame}


\begin{frame}[t]{Weak Law of Large Numbers}
Proof of WLLN:
\begin{itemize}
\item Remember: $\overline{X}_N\ = \frac{1}{N}\sum_{n=1}^N X_n$ \pause
\item Note that: $\Var{}{\overline X_n} = \frac{\Var{}{X}}{N} = \frac{c}{N}$ (we assume finite variance) \pause
\item By Chebyshev:
\begin{align}
P(|\overline X_n - \Exp{}{X}| > \epsilon) &\leq \frac{\sigma^2}{\epsilon^2} \\
&= \frac{c}{N\epsilon^2}
\end{align} \pause
\item For \emph{any} fixed $\epsilon$, $\lim_{N\to\infty} \frac{c}{N\epsilon^2} = 0$ \pause
\item $\implies \lim_{N\to\infty}\mathbb P\left(|\overline X_n - \mu| < \epsilon\right) = 1\qquad\qquad$ Done.
\end{itemize}
\end{frame}



\section{Generalisation Error Bounds}
\begin{frame}{LLN is a Detour}
\begin{itemize}
\item LLN ignores the size of the variance \pause
\item To prove LLN, we used a bound that \emph{did} depend on the size of the variance! \pause
\end{itemize}

\vspace{0.3cm}

\begin{center}
Can we use knowledge of the size of the variance \\ to say something more about generalisation error?
\end{center}
\end{frame}




\begin{frame}{Generalisation Error Bound}
A \emph{Generalisation Error/Loss Bound} is a procedure for computing a number $\epsilon$ from data that you sample form the world, such that \pause
\begin{itemize}
\item with high probability, \pause
\item the expected loss is below $\epsilon$. \pause
\end{itemize}
\begin{align}
\mathbb P\left(|L_{\text{test}} - \mathrm{ER}| > \epsilon\right) < \delta \\
\mathrm{ER} = \Exp{\pi(x, y)}{\ell(f(x; \vtheta^*), y)}
\end{align}
\end{frame}


\begin{frame}{Classification GEB}
\begin{itemize}
\item Consider Classification where $f : \mathcal X \to [0, 1]$. \pause
\item For \emph{testing}, we use 0-1 loss function (classification accuracy)
\begin{align}
\ell(f(x; \vtheta^*), y) = \begin{cases}
0 \qquad \text{if } \text{\texttt{int}}(f(x; \vtheta^*)) = y \\
1 \qquad \text{otherwise}
\end{cases}
\end{align} \pause
\item Remember $L_{\text{test}} = \frac{1}{N} \sum_{n=1}^N \ell(f(x; \vtheta^*), y)$
\item Remember $\Exp{\pi(x, y)}{L_{test}} = \mathrm{ER}$ \\ ($x = [x_1, x_2, \dots]$, and $y = [y_1, y_2, \dots]$).
\end{itemize}
\end{frame}

\begin{frame}{Chebyshev GEB}
Apply Chebyshev: \pause
\begin{align}
\mathbb P(|L_{\text{test}} - \mathrm{ER}| > \epsilon) &< \frac{\sigma^2}{\epsilon^2} \\
\uncover<3->{\sigma^2 &= \Var{\pi(x, y)}{L_{\text{test}}} \\
 &= \frac{1}{N}\Var{\pi(x, y)}{\ell(f(x; \vtheta^*), y)}}
\end{align} \pause \pause
Notice: $\Var{\pi(x, y)}{\ell(f(x; \vtheta^*), y)} < 0.25$! \pause
\begin{align}
\mathbb P(|L_{\text{test}} - \mathrm{ER}| > \epsilon) < \frac{0.25}{N\epsilon^2}
\end{align} \pause
\vspace{-0.8cm}
\begin{align}
\implies \mathbb P(\mathrm{ER} > L_{\text{test}} + \epsilon) < \frac{0.25}{N\epsilon^2}
\end{align}
{\tiny(Draw double-sided plot on board. $L_{\text{test}}$ is RV, and we only care about under-estimation of ER.)}
\end{frame}


\begin{frame}{Example Chebyshev GEB}
Q1: How accurate is our estimate of the expected loss?
\begin{itemize}
\item You train a NN on MNIST \pause 
\item Test error with $N=10000$ gives $L_{\text{test}} = 0.01$ \pause
\item Then Chebyshev gives us the guarantee that \pause
\begin{align}
\mathbb P(\mathrm{ER} > L_{\text{test}} + 0.03) < \frac{0.25}{N\cdot 0.03^2} = 0.0278 && \text{Pretty confident} \\
\mathbb P(\mathrm{ER} > L_{\text{test}} + 0.01) < \frac{0.25}{N\cdot 0.01^2} = 0.25 && \text{Not confident} \\
\mathbb P(\mathrm{ER} > L_{\text{test}} + 0.001) < \frac{0.25}{N\cdot 0.001^2} = 25 && \text{\emph{Vacuous}}
\end{align}
\end{itemize}
\end{frame}


\begin{frame}{How good is this?}
\begin{itemize}
\item We can guarantee with high probability that the classifier isn't an order of magnitude worse than $L_{\text{test}}$ indicates \pause
\item However bound is not tight enough to distinguish different methods, which often differ in accuracy by ±0.001 \pause
\item Probably \emph{very} pessimistic \pause
\item Bound holds for \emph{any} distribution with a maximum variance!
\end{itemize}
\end{frame}


\begin{frame}{Flipping bound round}
Q2: How big should our test set be, to get a certain accuracy?
\begin{gather}
\mathbb P(\mathrm{ER} > L_{\text{test}} + \epsilon) < \delta \\
\implies N > \frac{0.25}{\delta\epsilon^2}
\end{gather} \pause

\begin{itemize}
\item For $\epsilon = 0.001$, and $\delta = \frac{0.25}{N\epsilon^2} < 0.05$, we need $N > 5\cdot 10^6$!
\item For $\epsilon = 0.001$, and $\delta = \frac{0.25}{N\epsilon^2} < 0.01$, we need $N > 25\cdot 10^6$!
\item For $\epsilon = 0.01$, and $\delta = \frac{0.25}{N\epsilon^2} < 0.05$, we need $N > 50\cdot 10^3$!
\item For $\epsilon = 0.01$, and $\delta = \frac{0.25}{N\epsilon^2} < 0.01$, we need $N > 250\cdot 10^3$!
\end{itemize}
\end{frame}


\section{Assumptions and Quality of Bounds}

\begin{frame}{Hoeffding's inequality}
\begin{myblock}{}
For iid RVs $X_1, X_2, \dots$, such that $a < X_n < b$, $S_N = \frac{1}{N}\sum_n X_n$, and $t > 0$, we have
\begin{align}
\mathbb P\left(|S_N - \Exp{\pi}{S_N}| \geq t\right) \leq 2\exp\left(-\frac{2t^2N}{(b-a)^2}\right)
\end{align}
\end{myblock}

Proof not covered in course :)
\end{frame}



\begin{frame}{Hoeffding GEB}
Again, for classification
\begin{gather}
\mathbb P\left(\mathrm{ER} > L_{\text{test}} + \epsilon\right) \leq \delta \\
\implies N \geq \frac{\log(2\delta^{-1})}{2\epsilon^2}
%\delta = 2\exp\left(-\frac{2t^2N}{(b-a)^2}\right)
\end{gather} \pause
\begin{itemize}
\item For $\epsilon = 0.001$, and $\delta = \frac{0.25}{N\epsilon^2} < 0.05$, we need $N > 1.85\cdot 10^6$!
\item For $\epsilon = 0.001$, and $\delta = \frac{0.25}{N\epsilon^2} < 0.01$, we need $N > 2.65\cdot 10^6$!
\item For $\epsilon = 0.01$, and $\delta = \frac{0.25}{N\epsilon^2} < 0.05$, we need $N > 18.5\cdot 10^3$!
\item For $\epsilon = 0.01$, and $\delta = \frac{0.25}{N\epsilon^2} < 0.01$, we need $N > 26.5\cdot 10^3$!
\end{itemize} \pause
\begin{center}
Significant reduction compared to Chebyshev!
\end{center}
\end{frame}


\section{Conclusion}
\begin{frame}{Conclusion}
\begin{itemize}
\item Applying concentration inequalities (skill)
\item Can tell us accuracy of test set estimates
\item Concentration inequalities all relied on unbiased estimates
\item Variance determined accuracy
\end{itemize}
\end{frame}










\end{document}
%%% Local Variables: 
%%% mode: latex
%%% TeX-master: t
%%% End: 
