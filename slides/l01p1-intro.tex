%% Time-stamp: <2018-10-18 20:24:12 (marc)>
\documentclass[xcolor=x11names,compress,mathserif,handout]{beamer}

\newcommand{\hackspace}{\hspace{4.2mm}}
\newcommand{\showstudent}[1]{}
\newcommand\hmmax{0}
\newcommand\bmmax{0}





% talk/author information
\newcommand{\authorname}{Mark van der Wilk}
\newcommand{\authoremail}{m.vdwilk@imperial.ac.uk}
\newcommand{\authortwitter}{markvanderwilk}
\newcommand{\authoraffiliation}{
  Department of Computing\\Imperial
  College London}
\newcommand{\slidesettitle}{\imperialBlue{Mathematics for Machine Learning: Course Overview}}
\newcommand{\footertitle}{Course Overview}
\newcommand{\location}{Imperial College London}
\newcommand{\talkDate}{October 3, 2022}



\date{\imperialGray{\talkDate}}




% load defaults
\usepackage{../includes/MarkMathCmds}
\input{../includes/header.tex}



\input{../includes/titlepage.tex}
\linespread{1.2}



\begin{frame}{Goal}
In this course you will learn mathematical principles used to
\begin{center}
\Large 1. \emph{create} ML algorithms. \\ \pause
2. \emph{analyse} ML algorithms.
\end{center}
\pause
Key skills
\begin{itemize}
\item Probability meets linear algebra, i.e.~multiple variables
\item Linear algebra and statistics underlying learning algorithms
\item Probability to analyse learning algorithms
\end{itemize}

Follows on from:
\end{frame}

% \begin{frame}
% \begin{itemize}
% \item Course book: Mathematics for Machine Learning (\url{https://mml-book.github.io})
% \item We have to assume knowledge of vector spaces \& linear algebra (ch 2-3)
% \item Problem driven: We introduce mathematical techniques in the context of fundamental ML methods.
% \end{itemize}
% \end{frame}


\begin{frame}{Prerequisites}
This course follows on from:
\begin{itemize}
\item \texttt{40016} - Y1 - Calculus
\item \texttt{40017} - Y2 - Linear Algebra
\item \texttt{50008} - Y2 - Probability \& Statistics
\end{itemize}

\vspace{0.5cm}

An incomplete collection of skills that will be assumed:
\begin{itemize}
\item Linear Algebra (e.g.~change of basis, eigenvectors)
\item Differentiation \& Integration
\item Probability and basic stats (maximum likelihood)
\end{itemize}

\vspace{0.5cm}

We collected exercises to support you in revising this.
\end{frame}



\begin{frame}{Logistics: In-person teaching}
Two lecturers: Myself, and Dr Yingzhen Li.

Teaching schedule:
\begin{itemize}[<+->]
\item MvdW: Weeks 2-3 (Mon), 5 (Fri)-7 (Mon).
\item YL \hspace{0.6cm}: Weeks 3 (Fri)-5 (Mon), 7 (Mon)-8.
\item In-person teaching. Lectures are recorded, but best to show up.
\item In-person TA sessions on Fri 10-11am.
\end{itemize}
\end{frame}


%%%%%%%%%%%%%%%%%%%%%%%%%%%%%%%%%%%%%%%%
\begin{frame}{Coursework}
Two courseworks.
\begin{itemize}
\item Coding exercise, designed to put the theory into practice.
\item Code submission to LabTS, where it will be automatically graded by unittests.
\item Feedback provided by TAs.
\item Schedule will be online shortly.
\end{itemize}
\end{frame}


\begin{frame}{Exercise sheet}
\begin{itemize}
\item Unassessed but crucial for your practice.
\item TAs are here \textbf{every week} to provide feedback.
\item But, your responsibility to instigate.
\end{itemize}
\end{frame}




\begin{frame}{TA sessions}
\begin{itemize}
\item Your opportunity ask questions, discuss exercises, get feedback.
\item This is \emph{student led}: up to you to make the most of it. Suggestion:
\begin{itemize}
\item Discuss specific questions about the course material.
\item Go through (selection of) exercises. You explain your solution to the TAs. TA gives feedback on your solution.
\item Discuss differences to the solution provided.
\item Discuss steps you are uncertain about (e.g.~why is a step needed).
\end{itemize}
\item You will need to sign up to a specific TA beforehand (to be communicated over EdStem).
\end{itemize}
\end{frame}




\begin{frame}{EdStem}
You can also ask questions on EdStem, and TAs or lecturers will respond.
\end{frame}





\begin{frame}{Course overview}
In this course, we will consider two machine learning problems:
\begin{itemize}
\item Supervised learning
\item Unsupervised learning
\end{itemize}

\vspace{0.3cm}

We will teach you the mathematics needed to \emph{implement} and \emph{analyse} the methods, e.g.:

\begin{itemize}
\item Linear Regression
\begin{itemize}
\item Differentiation, Optimisation (implementation)
\item Probability and Statistics (analysis)
\end{itemize}
\end{itemize}

We care catering to students with a wide variety of backgrounds. Some need to catch up, so there is a lot of material. Focus on the skills needed for implementation if it is too much.
\end{frame}


\begin{frame}{Course materials}
\begin{itemize}
\item The \textbf{only} link you need for material: \texttt{https://scientia.doc.ic.ac.uk/2223/modules/70015}
\item This links to all our materials on GitHub:
\texttt{https://github.com/markvdw/mml-autumn-2022}
\item All \LaTeX\ sources are on GitHub too.
\item Please do suggest improvements to exercise solutions, fix typos, etc. Just fork and submit a PR.
\item For those who contribute, there will be cake.
\end{itemize}
\end{frame}










\end{document}
%%% Local Variables: 
%%% mode: latex
%%% TeX-master: t
%%% End: 
